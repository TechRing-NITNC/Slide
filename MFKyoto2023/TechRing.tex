\documentclass[dvipdfmx]{beamer}
%テーマ
\usetheme[numbering=counter,shape=square]{LightTheme}
%パッケージ
\usepackage{pxjahyper}
%フォント
\renewcommand{\kanjifamilydefault}{\gtdefault}

%タイトル設定
\title{TechRingにかける想い}
\subtitle{学科を超えて}
\author[TechRing]{奈良高専TechRing}
\date[2023/04/30]{2023/04/30}

\begin{document}

\maketitle

\begin{frame}{高専どんなところ?}

  \begin{large}
    国立高等専門学校の特徴
  \end{large}
  \begin{itemize}
    \setlength{\itemsep}{2mm}
    \item 5年制、一貫のカリキュラム
    \item 電気電子系や機械系に限らず、物質や建築、商船もある
    \item 高専祭という文化祭での文化発表が活発
    \item ロボコンやデザコン、プロコンなどのコンテストが充実
    \item 高専カンファレンスや様々な技術コミュニティでの交流
  \end{itemize}

  \vspace{3mm}
  \begin{large}
    奈良高専の特徴
  \end{large}
  \begin{itemize}
    \setlength{\itemsep}{2mm}
    \item 来年で創設60周年を迎える
    \item 機械、電気、電子制御、情報、物質科学の5学科
    \item 最寄り駅からバスで20分ほど
    \item 女子枠推薦がある
  \end{itemize}
\end{frame}

\begin{frame}{起業家工房について}
  国の進める高専スタートアップ事業、起業家工房設立について話す
\end{frame}

\begin{frame}{TechRingとは?}
  TechRingとは何か、自己紹介
\end{frame}

\begin{frame}{TechRingの目的}
  TechRingの目指す目的を紹介
\end{frame}

\begin{frame}{TechRingの担う学内活動 その1}{イベント開催}
  奈良高専生に向けてイベントを開催する話\\
  LTイベントや何か製作イベントなど
\end{frame}

\begin{frame}{TechRingの担う学内活動 その2}{技術交流}
  5団体でノウハウや活動の共有をしたり、\\
  コンテストの共同参加やその他技術交流に関する話
\end{frame}

\begin{frame}{TechRingの担う学内活動 その3}{共同制作}
  技術交流を通して共同で制作をする話\\
  共同制作したものを外部公開する話
\end{frame}

\begin{frame}{TechRingを創る仲間たち その1}{機械研究会}
  機械研究会にフォーカス\\
  活動の紹介と出展の紹介を
\end{frame}

\begin{frame}{TechRingを創る仲間たち その2}{MeCafe}
  MeCafeにフォーカス\\
  活動の紹介と出展の紹介を
\end{frame}

\begin{frame}{TechRingを創る仲間たち その3}{電気技術研究会}
  電気技術研究会にフォーカス\\
  活動の紹介と出展の紹介を
\end{frame}

\begin{frame}{TechRingを創る仲間たち その4}{システム開発研究会}
  システム開発研究会にフォーカス\\
  活動の紹介
\end{frame}

\begin{frame}{TechRingを創る仲間たち その5}{情報処理研究会}
  情報処理研究会にフォーカス\\
  活動の紹介と出展の紹介を
\end{frame}

\begin{frame}{TechRingのこれから}
  TechRingの今後の活動を紹介
\end{frame}

\begin{frame}
  \begin{Large}
    ご清聴、ありがとうございました。
  \end{Large}
\end{frame}

\end{document}